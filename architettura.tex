\chapter{Architettura software}


\section{Architettura generale}

Il progetto è stato realizzato seguendo un'architettura software client-server (figura \ref{fig:arch}). In questa architettura i componenti principali sono due: il client rappresentato dall'applicazione iPad oggetto di questo documento e il lato server scomposto in \emph{middleware} e nei sistemi messi a disposizione dal cliente (l'istituto bancario).

Il middleware opera da intermediario tra i servizi bancari e l'applicazione ed  ha il compito di ricevere le richieste dal client, di sottomettere a sua volta queste richieste ai servizi lato banca e di ritornare i dati al client opportunamente codificati.

Le richieste verso il middleware (ospitante un webserver \emph{Apache}) vengono eseguite seguendo un'architettura \emph{RESTFUL}  che prevede richieste \emph{HTTP} (POST, GET, PUT, DELETE). Le risposte fornite dal middleware verso il client sono costruite in formato \emph{JSON}.


\begin{figure}[!htbp]
\centering
\includegraphics[scale=0.60]{architettura/architect.png}
\caption{Architettura client-server}
\label{fig:arch}
\end{figure}

\section{Architettura dell' applicazione}

L'applicazione è stata progettata usando il design pattern \emph{MVC} (Model-View-Controller, figura \ref{fig:mvc}). In tale pattern ogni oggetto dell'applicazione assolve uno dei seguenti ruoli:
\begin{itemize}
 \item Model: gli oggetti di tipo model hanno il compito di incapsulare i dati specifici di un'applicazione e di definire le logiche per modificare e processare tali dati
 \item View: sono gli oggetti dell'applicazione visibili all'utente. Tali oggetti hanno il compito di visualizzare i dati dell'applicazione e di permetterne l'interazione (esempio la modifica)
 \item Controller: è un oggetto che opera da intermediario tra la view e uno o più model dell'applicazione. Tali oggetti hanno quindi la funzione di comunicare i cambiamenti tra le view e i model, e viceversa.
\end{itemize}

L'utilizzo del pattern MVC permette una migliore estensione del codice, di costruire componenti riusabili e migliorare la definizione delle interfacce.

\begin{figure}[!htbp]
\centering
\includegraphics[scale=0.70]{architettura/mvc.png}
\caption{Schema del pattern MVC}
\label{fig:mvc}
\end{figure}

Nei prossimi paragrafi verranno descritte le classi principali del progetto.

\section{Diagramma delle classi}

\subsection{Framework e librerie esterne}
Oltre agli strumenti messi a disposizione dalla Apple sono stati utilizzati framework esterni (open source) che hanno permesso di ridurre i tempi di sviluppo, di seguito due esempi:
\begin{itemize}
 \item CorePlot: è un framework per il disegno di grafici in applicazioni iOS, fornsisce strumenti per il disgeno e la visualizzazione dei dati in 2D
 \item JNKeychain: wrapper sulle funzioni del \emph{KeyChain}, permette il salvataggio e il recupero dei dati in maniera sicura dei dati confindenziali
\end{itemize}


\subsection{Business Logic}
Di seguito sono rappresentate una parte delle classi della \emph{business logic}.
\begin{figure}[!htbp]
\centering
\includegraphics[scale=0.60]{architettura/businessLogicClass.png}
\caption{Classi della business logic}
\label{fig:businessLogic}
\end{figure}

\subsection{Connessione ai servizi}
\label{parag:networking}
La parte relativa alla gestione delle chiamate ai servizi è stata realizzata implementando classi di tipo \emph{singleton}. Tali classi estendono la libreria di terze parti \texttt{AFNetworking} che opera da wrapper sulle tecnologie di comunicazione messe a disposizione dall'SDK iOS (Foundation URL Loading System). 

Utilizzare una libreria come AFNetworking ha permesso di ridurre i tempi di sviluppo e di sfruttare le potenzialità offerte dalle sue API come: la serializzazione e deserializzazione dei dati, la possibilità di effettuare richieste HTTP asincrone, una gestione efficiente delle code, la gestione di sessioni, ecc\dots.  

\begin{figure}[!htbp]
\centering
\includegraphics[scale=0.70]{architettura/networkingClass.png}
\caption{Diagramma delle classi per le connessioni di rete}
\end{figure}

\subsubsection{Autorizzazione}

La classe singleton \texttt{AuthHTTPClient} ha il compito di effettuare il login ai servizi e di ottenere le credenziali di accesso alle risorse protette. Per l'autorizzazione e  l’accesso alle risorse offerte dai servizi bancari è stato utilizzato il protocollo open \emph{OAut 2.0} (Open Authorization).

Nella classica autenticazione client-server, il client ottiene l’accesso alle risorse protette, ospitate sul server, richiedendo l’autenticazione ad esso attraverso la sottomissione delle credenziali del proprietario delle risorse. Quindi per fornire ad ogni applicazione l’accesso alle risorse protette, il proprietario di tale risorse deve condividere le sue credenziali con queste applicazioni. Questo può portare a diversi problemi di sicurezza e limitazioni (argomenti non trattati in questo documento).
 
Nel protocollo OAuth viene fatta una separazione tra il client e il proprietario delle risorse aggiungendo un ulteriore layer che si occupa dell’autorizzazione delle applicazioni di terze parti.
Invece di usare le credenziali del proprietario delle risorse il client utilizza un \emph{access token} (una stringa che indica un preciso scope, tempo di vita, ecc\dots).
Questo access token è fornito da un \emph{Authorization Server} previo permesso del proprietario delle risorse.

Le entità coinvolte in questo protocollo sono:



\begin{itemize}
  \item Resource server: il sistema in cui sono memorizzate le risorse
  \item Resource owner: il proprietario delle risorse, colui che possiede le credenziali sul Resource server
  \item Protected resource: dati controllati dal Resource server
  \item Client: un’applicazione che usa OAuth per accedere alle risorse ospitate sul Resource server, sotto la delega del Resource owner
\end{itemize}



Le figure \ref{fig:oauth} e \ref{fig:refresh} mostrano lo scambio dei dati tra un Client, un Authorization Server e un Resource Server nel caso in cui viene richiesto per la prima volta l'autenticazione e nel caso in cui viene generato un nuovo token a partire da un \emph{Refresh Token}.

\begin{figure}[!htbp]
\centering
\includegraphics[scale=0.50]{architettura/token.png}
\caption{Nel diagramma è mostrato il flusso di richiesta di un token per l'accesso alle risorse. Il client richiede al Resource Owner il permesso per l'autenticazione. Il Resource Owner invia al Client il permesso rappresentato come un codice che segue le specifiche del protocollo. Il Client richiede all'Authorization Serve un access token passandogli il codice di autorizzazione. L'Authorization Server autentica il client e ritorna il token. Il Client utilizzerà tale token per accedere alle risorse protette memorizzate nel Resource Server. }
\label{fig:oauth}
\end{figure}


\begin{figure}[!htbp]
\centering
\includegraphics[scale=0.40]{architettura/refreshToken.png}
\caption{Il diagramma mostra il caso in cui la richiesta ad una risorsa protetta avvenga passando al Resource Server un token valido e il caso in cui il token corrente sia non valido. In quest'ultimo caso viene generato un nuovo token a partire dal \emph{refresh token}. }
\label{fig:refresh}
\end{figure}

% i servizi sono separati da quelli di autenticazione perchè gli  standard oauth richiedono la separazione i sottodimini diversi.
\newpage
\subsection{Caching dei dati}
Per migliorare le prestazioni e limitare il traffico di rete generato durante l'utilizzo dell'applicazione è stato implementato un meccanismo di caching dei dati che consente di salvare in memoria primaria i dati necessari all'applicazione durante una sessione di utilizzo. 

\begin{figure}[!htbp]
\centering
\includegraphics[scale=0.70]{architettura/cacheClass.png}
\caption{Diagramma delle classi per il caching dei dati}
\end{figure}

I controller possono richiamare l'oggetto singleton \texttt{SharedDataHandler} e attraverso esso salvare in variabili temporanee i dati che non necessitano di essere riscaricati dalla rete durante una sessione .

\subsection{Autenticazione e sicurezza dei dati}
L'applicazione prevede un sistema di autenticazione ai servizi basata su livelli. In particolare sono previsti tre livelli di autenticazione:
\begin{itemize}
 \item autenticazione di livello zero: è il livello associato a un utente che non ha ancora effettuato il login
 \item autenticazione di primo livello: è il livello associato a un utente che ha precedentemente effettuato il login, ha scelto di ricordare le credenziali ed ha momentaneamente sospeso l'utilizzo dell'applicazione. Questo livello permette all'utente di accedere a un gruppo ristretto di funzionalità nel momento in cui riprenderà l'utilizzo dell'applicazione
 \item autenticazione di secondo livello: è il livello associato a un utente che ha precedentemente effettuato il login e che permette l'accesso a tutte le funzionalità previste dall'applicazione. Questo livello permane fino a quando l'utente non richiede esplicitamente il logout o sospende l'applicazione
\end{itemize}

\begin{figure}[!htbp]
\centering
\includegraphics[scale=0.70]{architettura/stati.png}
\caption{Diagramma degli stati di autenticazione}
\end{figure}


L'autenticazione ai servizi è a carico delle classi descritte nel paragrafo \ref{parag:networking}. Questa operazione restituisce in output i dati confidenziali dell'utente che saranno gestiti dalle classi adibite alla sicurezza.

A tal fine è stato implementato un oggetto singleton, il \texttt{SecurityManager}, che ha la responsabilità di stabilire il livello di autenticazione in risposta agli eventi che occorrono all'interno dell'applicazione e di gestire il salvataggio e il recupero dei dati confidenziali dell'utente (come le informazioni sul token, sul numero utente, ecc\dots). Tale classe utilizza la libreria di terze parti \texttt{JNKeychain} che estende le funzioni messe a disposizione dell'SDK iOS, in particolare quelle relative al \emph{Keychain}. 

\begin{figure}[!htbp]
\centering
\includegraphics[scale=0.70]{architettura/securityClass.png}
\caption{Diagramma delle classi relative alla sicurezza}
\end{figure}
\newpage
\subsection{Controller principali}
Di seguito è riportato il diagramma delle classi rappresentanti i controller principali creati durante la progettazione e lo sviluppo. Per semplicità di esposizione sono stati rappresentati solo i nomi dei controller e le relazioni tra essi.
\newline

A titolo esemplificativo nel diagramma si può notare che il \texttt{BaseViewController} è il controller principale dal quale gli altri controller ereditano le funzionalità e proprietà generiche. Tale controller si occupa della creazione e visualizzazione degli  elementi grafici comuni a tutte le interfacce (ad esempio la navigation bar personalizzata) e delle logiche necessarie ad ogni controller che lo specializza.

In particolare uno dei compiti principali di questo controller è quello di creare il controller che gestisce l'interfaccia e la logica per il login ai servizi (\texttt{LoginViewController}) e di implementare i metodi nel  protocollo \texttt{LoginViewControllerProtocol} dichiarato dallo stesso LoginViewController. Sarà quindi compito del LoginViewController  lanciare il processo di login e comunicare al BaseViewController l'esito dell'operazione attraverso il protocollo descritto precedentemente; così facendo il BaseViewController sarà in grado di effettuare l'aggiornamento dello stato dell'interfaccia.


 \newpage
% \begin{landscape}
\begin{figure}[!htbp]
\centering
\includegraphics[scale=0.65]{architettura/controllersNew.png}
\caption{Diagramma delle classi controller}
\end{figure}
% \end{landscape}



\section{Diagramma di sequenza}

Di seguito è riportato un esempio di diagramma di sequenza relativo allo scenario in cui un utente richieda la lista dei conti e delle carte a lui registrate; in tale diagramma saranno evidenziati le sequenze di messaggi scambiati tra le entità che entrano in gioco durante il processo di recupero e visualizzazione delle informazioni. 
 \begin{landscape}
\begin{figure}[!htbp]
\centering
\includegraphics[scale=0.65]{architettura/sequence.png}
\caption{Diagramma si sequenza per la visualizzazione della lista conti e carte}
\end{figure}
\end{landscape}