% !TEX encoding = UTF-8
% !TEX program = pdflatex
% !TEX spellcheck = it_IT

\documentclass[LaM,binding=0.6cm, oneside]{sapthesis}

\usepackage{microtype}
\usepackage[italian]{babel}
\usepackage[utf8]{inputenx}
\usepackage{booktabs}
\usepackage{tabularx}
\usepackage{graphicx}
\usepackage{dcolumn}
\usepackage{hyperref}
\usepackage{longtable}
\usepackage{enumerate}
\usepackage{lscape}
\usepackage{amsmath}
\usepackage{mathtools}
% \usepackage{pdflscape}
% \usepackage{geometry}
\usepackage[]{algorithm2e}
\hypersetup{pdftitle={Titolo della tesi},pdfauthor={Mario Greco}}
% \usepackage[landscape]{geometry}
% Remove in a normal thesis
\usepackage{lipsum}
\usepackage{curve2e}
\usepackage{setspace}
\usepackage{xcolor}
\usepackage{listings}
\definecolor{gray}{gray}{0.4}
\newcommand{\bs}{\textbackslash}

% Commands for the titlepage
\title{Progettazione e sviluppo di un'applicazione di Home Banking in ambiente iOS}
\author{Mario Greco}
\IDnumber{984280}
\course{Informatica}
\courseorganizer{Facoltà di Ingegneria dell'Informazione, Informatica e Statistica}
\submitdate{2013/2014}
\copyyear{2014}
\advisor{Prof. Emanuele Panizzi}
% \advisor{Dr. Nome Cognome}
\coadvisor{Dott. Raffaele Gitto}
\authoremail{mrgreco3@gmail.com}

% \examdate{18 Marzo 2014}
% % \examiner{Prof. Nome Cognome}
% % \examiner{Prof. Nome Cognome}
% % \examiner{Dr. Nome Cognome}
\versiondate{\today}


\DeclareCaptionFont{white}{\color{white}}
\DeclareCaptionFormat{listing}{%
  \parbox{\textwidth}{\colorbox{gray}{\parbox{\textwidth}{#1#2#3}}\vskip-4pt}}
\captionsetup[lstlisting]{format=listing,labelfont=white,textfont=white}
\lstset{frame=lrb,xleftmargin=\fboxsep,xrightmargin=-\fboxsep}
\begin{document}
\doublespace 
\frontmatter

\maketitle

\dedication{Dedicato a chi non ha mai mollato}

\begin{abstract}
Il lavoro presentato in questo documento (realizzato come tirocinio presso l'azienda NTT Data Italia) descrive le diverse fasi di progettazione e di sviluppo di un'applicazione mobile, per dispositivi Apple iPad,  commissionata da un'importante istituto finanziario.

L'applicazione avrà lo scopo sia di permettere all'utente finale (il titolare di un conto bancario) di usufruire dei principali servizi messi a disposizione dall'istituto bancario sia di offrire soluzioni sempre più innovative, a portata di mano e vicine alle reali necessità dell'utente.

% Il progetto ha lo scopo di abbracciare diverse figure di utenti che spaziano da
% 
% JobNavigator si rivolge a una figura di utente che può spaziare dal datore di lavoro alla ricerca di personale, al disoccupato o a quanti desiderino trovare una migliore sistemazione lavorativa. Non solo, infatti con JobNavigator chiunque può diventare parte attiva del sistema semplicemente segnalando le offerte di lavoro di cui è venuto a conoscenza: in questo modo si danno maggiori opportunità a quanti sono impegnati nella difficile e continua ricerca di un impiego. 

Nell'esposizione è stata data enfasi alle scelte progettuali effettuate, descrivendo i problemi incontrati durante il ciclo di vita del software e le relative soluzioni adottate.

% Il software JobNavigator è attualmente in corso di pubblicazione sul servizio Apple \emph{App Store}. 
\end{abstract}

\begin{acknowledgments}
Ringrazio anzitutto il  professor Emanuele Panizzi, Relatore: senza il suo supporto e la sua guida sapiente questa tesi non esisterebbe. 
\\\\
Ringrazio inoltre il Dott. Raffaele Gitto, Co-Relatore, e tutti i colleghi della NTT Data per tutto quello che hanno fatto per me durante il periodo di stage.
\\\\
Un sentito grazie a tutti gli amici e i colleghi universitari che mi hanno supportato e sopportato durante questi anni, in particolare: Domenico, Federico, Flavia, Giovanni. 
\end{acknowledgments}

\tableofcontents

% % Do not use the starred version of the chapter command!
% \chapter{Capitolo non numerato}
% 
% In this manual you can skip the gray text because it is just dummy text.%
% \footnote{This is a footnote.}
% 
% \textcolor{gray}{\lipsum[1-22]}
% 
% 
% \section*{Paragrafo non numerato}
% 
% In this manual you can skip the gray text because it is just dummy text.
% 
% \textcolor{gray}{\lipsum[1-22]}

 \large


\mainmatter


\chapter{Introduzione al progetto}

\section{Descrizione dell'applicazione}



\section{Primi requisiti funzionali}
Di seguito sono riportati i primi requisiti funzionali ricavati dal processo di analisi delle richieste del committente, ottenute durante le interviste iniziali del progetto. Tali requisiti definiscono le funzionalità e i servizi 
offerti dal sistema da realizzare.

\begin{center}
       \captionof{table}{Primi requisiti funzionali}

    \begin{tabular}{p{6cm}|p{8cm}}

    \toprule
    \multicolumn{1}{c}{\textbf{Requisito funzionale}} &
    \textbf{Descrizione}\\

    \midrule
    Login ai servizi & Permettere all'utente l'accesso ai servizi bancari mediante credenziali \\
    Visualizzazione riepilogo conti e carte & Fornire opportune viste di riepilogo dei prodotti posseduti da un utente (esempio: conti correnti, carte di credito, ecc\dots)\\
    Visualizzazione storico saldi & Rappresentare tramite grafici dello storico saldi di un prodotto\\
    Riepilogo lista movimenti & Recuperare e visualizzare la lista dei movimenti di un determinato prodotto\\
    Filtraggio lista movimenti & Permettere il recupero dei movimenti in base un certo arco temporale\\
    Disporre operazioni bancarie & Permettere l'esecuzione di dispositive bancarie (esempio: bonifici, giroconti, ecc\dots) \\
    Interazione con i social network & Consentire la visualizzazione dei contenuti social messi a disposizione del cliente\\
    Messaggistica & Permettere la lettura delle comunicazioni ricevute  \\
    \bottomrule

    \end{tabular}
        \label{tab:requisiti_iniziali}

\end{center}



\section{Primi mockup, wireframe e prototipi}
\subsection{Mockup}
La realizzazione di mockup grafici ha permesso di offrire al cliente una prima rappresentazione visiva dei requisiti funzionali iniziali. Di seguito sono riportate alcuni di questi mockup relativi allo stadio iniziale del progetto.

\begin{figure}[!htbp]
\centering
\includegraphics[scale=0.7]{immagini_mockup/home.png}
\caption{Home contenuti social e pannello login}
\end{figure}

\begin{figure}[!htbp]
\centering
\includegraphics[scale=0.7]{immagini_mockup/bolle.png}
\caption{Riepilogo conti e carte}
\end{figure}

\begin{figure}[!htbp]
\centering
\includegraphics[scale=0.7]{immagini_mockup/timeline.png}
\caption{Timeline movimenti}
\end{figure}

\begin{figure}[!h]
\centering
\includegraphics[scale=0.7]{immagini_mockup/bonifico.png}
\caption{Operazione bonifico}
\end{figure}

\newpage
\subsection{Wireframe}
Parallelamente ai mockup e durante tutto il ciclo di vita del software sono stati realizzati e raffinati i wireframe. 

I wareframe forniscono una rappresentazione strutturale di un applicazione software e permettono di individuare le dinamiche del progetto in termini di usabilità ed utilizzo pratico, i punti critici e quelli che richiedono uno sviluppo più accurato o miglioramenti.

Di seguito sono mostrate alcune delle immagini di uno dei primi wireframe realizzati:

\begin{figure}[!htbp]
\centering
\includegraphics[scale=1.0]{primo_wireframe/miasituazione.png}
\caption{Riepilogo conti e carte}
\end{figure}
\begin{figure}[!htpb]
\centering
\includegraphics[scale=1.0]{primo_wireframe/timeline2.png}
\caption{Timeline movimenti}
\end{figure}
\begin{figure}[!htbp]
\centering
\includegraphics[scale=1.0]{primo_wireframe/bonifico2.png}
\caption{Dettaglio bonifico}
\end{figure}

\newpage
\subsection{Prototipi}
Un'altro passo fondamentale durante la fase iniziale del progetto è stato la realizzazione di prototipi.

Un prototipo è un modello approssimato del sistema che si sta realizzando e che simula o esegue solo una parte delle funzioni del sistema finale.
La prototipizzazione permette di:

\begin{itemize}
  \item tenere il design centrato sull’utente 
  \item sperimentare design alternativi
  \item ottenere feedback rapidi sul progetto
  \item trascurare dettagli secondari (come qualità del codice, efficienza, ecc\dots) 
  \item valutare l'usabilità
  \item ridurre i rischi di un progetto permettendoci di mettere prima a fuoco alcune caratteristiche del sistema e capire se sono adeguate o meno
\end{itemize}

Durante le prime settimane del progetto sono stati quindi realizzati dei prototipi contenenti funzionalità \emph{stub}\footnote{Funzionalità che simulano il comportamento  del sistema restituendo valori accettabili in un ipotetico scenario reale.} descritte dalla tabella \ref{tab:requisiti_iniziali}. Tali prototipi sono stati successivamente messi a disposizione del cliente e testati su device Apple iPad, portando alla raccolta e valutazione dei primi feedback sull'utilizzo del software.

\chapter{Metodologie usate e raffinamenti successivi}

\section{Metodologia Agile}

L'intero ciclo di vita del software è stato gestito adottando una metodologia \emph{Agile}.

I metodi Agile sono tali da coinvolgere il più possibile il committente, dando quindi vita a un processo di tipo adattativo: cioè che si adatta alle esigenze del cliente, che possono cambiare durante lo sviluppo.
L'Agile è un processo costituito da finestre di tempo limitate (2-4 settimane) chiamate iterazioni, le quali sono a loro volta scomposte nelle fasi di progettazione, di sviluppo e di test.

\begin{figure}[!htbp]
\centering
\includegraphics[scale=0.5]{immagini/agile.png}
\caption{Iterazioni e fasi della metodologia Agile}
\end{figure}

Il progetto è quindi suddiviso in singoli componenti indipendenti dalle funzionalità così da poterne analizzare e valutare i costi e i tempi. Ogni iterazione conterrà quindi tutto ciò che è indispensabile per rilasciare un piccolo incremento nelle funzionalità del software e sono tali da essere soggette a modifiche al fine di migliorare l'architettura finale del software.

\section{Requisiti funzionali finali}

Una progettazione iterativa e focalizzata sull'utente ha permesso quindi di ampliare e raffinare i requisiti fino ad ottenere  un risultato sempre più vicino alle richieste e alle necessità del committente.  

Per brevità di seguito sono riportati solo alcuni dei requisiti finali del progetto (che vanno ad aggiungersi o a integrare quelli della tabella \ref{tab:requisiti_iniziali}), per comodità raccolti in categorie.

\begin{center}
    \captionof{table}{Requisiti funzionali finali}

    \begin{longtable}{p{6cm}|p{8cm}}

    \toprule
    \multicolumn{1}{c}{\textbf{Requisito funzionale}} &
    \textbf{Descrizione}\\

    \midrule
    Accesso ai servizi & \begin{itemize}
                          \item Permettere l'accesso ai servizi attraverso credenziali
                          \item Permette l'accesso veloce a un sottoinsieme di funzioni salvando le credenziali
                          \item Permettere recupero Codice Cliente
                         \end{itemize}\\\\

    Riepilogo movimenti & \begin{itemize}
                           \item Rappresentare i movimenti di un conto attraverso una timeline
                           \item Mostrare i movimenti di un conto in una lista
                           \item Visualizzare una scheda di dettaglio per un movimento selezionato
                          \end{itemize}\\\\
    Filtro movimenti & Offrire la possibilità di filtrare i movimenti per data e per tipo (entrate, uscite, tutti)\\\\
    Riepilogo conti e carte & \begin{itemize}
				\item Visualizzare i prodotti di un utente attraverso bubble contenenti informazioni di riepilogo
				\item Visualizzare i prodotti di un utente attraverso una lista
				\item Permettere la selezione di un determinato prodotto
                              \end{itemize}\\\\
    Filtro conti e carte & Permettere la visualizzazione o meno di certi prodotti mediante un filtro\\\\
    Storico saldi & Visualizzare in un grafico a bolle lo storico dei saldi di un prodotto per mesi o settimane \\\\
    Disporre operazioni bancarie & \begin{itemize}
				      \item Fornire un form per la compilazione di dispositive 
				      \item Permettere la riproposizione di dispositive già effettuate
				    \end{itemize}\\\\
    Riepilogo dispositive effettuate & \begin{itemize}
                                        \item Visualizzare le ultime $n$ dispositive in una lista ordinata
                                        \item Visualizzare le ultime $n$ dispositive in un grafico a bolle
                                        \item Permettere il salvataggio delle dispositive effettuate in una lista di preferiti
                                       \end{itemize}\\\\
    Preferiti & Permettere la visualizzazione e la modifica di una lista di dispositive scelte come preferite\\\\\
    Rubrica & Permettere l'accesso in lettura e scrittura alla rubrica del dispositivo per poter salvare o recuperare numeri telefonici\\\\
    
    Help e gestore & \begin{itemize}
                      \item Offrire una sezione interna al software in cui l'utente può richiedere appuntamenti con un consulente
                      \item Mettere a disposizione una sezione di help e numeri utili
                     \end{itemize}\\\\

    Integrazione social network & \begin{itemize}
                                   \item Visualizzare i contenuti dei canali social offerti dal committente
                                   \item Offrire la possibilità di condividere i contenuti sui diversi social network
				   \item Integrare un player video per la visualizzazione di filmati relativi alle attività del committente
                                  \end{itemize}\\
     Browsing interno & Permettere la navigazione tra contenuti web attraverso un browser interno all'applicazione\\\\
     Notifiche Push & Permettere la ricezione di notifiche push\\\\
     Messaggistica		 & \begin{itemize}
                                   \item Visualizzare i messaggi ricevuti dai servizi
                                   \item Permettere la gestione dei messagi (esempio: cancella, segna come letto, ecc\dots)
                                  \end{itemize}\\
    \bottomrule

    \end{longtable}
\end{center}

Sono inoltre elencati alcuni dei requisiti non funzionali:

\begin{center}
    
    \captionof{table}{Requisiti non funzionali}

    \begin{tabular}{p{6cm}|p{8cm}}

    \toprule
    \multicolumn{1}{c}{\textbf{Requisito non funzionale}} &
    \textbf{Descrizione}\\

    \midrule
    Comunicazione sicura & \begin{itemize}
                            \item Garantire la comunicazione sicura tra client e server
                            \item Garantire che i dati salvati all'interno del device siano protetti
                           \end{itemize}\\\\
    Ottimizzazione flusso dati & Limitare il più possibile il consumo dei dati\\\\
    Efficienza & Garantire uso efficiente delle risorse offerte dal device (cpu, batteria, ecc\dots)\\

    \bottomrule

    \end{tabular}
%     \label{tab:requisiti_non_funz}

\end{center}

\section{User Experience e usabilità}

Oltre al consolidamento dei requisiti, le fasi di analisi e test (racchiuse in ogni iterazione della metodologia Agile) hanno permesso una graduale evoluzione del progetto, partito da uno stadio prototipale, anche sotto il punto di vista dell'usabilità e della user experience.

I test e i collaudi sono stati eseguiti presentando il progetto su dispositivi Apple iPad a personale selezionato (per policy interne i collaudi sono avvenuti con personale interno all'azienda cliente) e sono stati discussi, raccolti e valutati i feedback ottenuti durante l'utilizzo del software.
Questo approccio ha consentito l'immediata valutazione della qualità del software prodotto e di concentrarsi immediatamente sull'iterazione successiva per migliorarlo.

\subsection{iOS Human Interface Guidelines}

Le \emph{GUI} (graphic user interface) sono state realizzate considerando le linee guida fornite da Apple nelle \emph{Human Interface Guidelines}. Di seguito sono elencate alcune delle linee chiave adottate:

\begin{itemize}
 \item Integrità estetica: rappresenta quanto l'apparenza di un'interfaccia è coerente con le azioni e le funzionalità che offre
 \item Consistenza: permettere all'utente di trasferire le proprie abilità e conoscenze da un'interfaccia di un'applicazione all'altra 
 \item Feedback: fornire un feedback all'utente permette di avere una risposta immediata di un'azione intrapresa sull'interfaccia 
\end{itemize}

Oltre ai componenti standard offerti dall'\emph{SDK} iOS sono stati realizzati dei componenti grafici custom che hanno migliorato l'esperienza d'uso rispettando al tempo stesso le linee guida e il know how medio posseduto dalle persone che utilizzano dispositivi mobili.  

\subsection{Esempi di GUI e raffinamenti successivi}

\subsubsection{Colori}
Inizialmente si era scelto di utilizzare un tema di colori focalizzato sullo stesso colore del brand del committente e degli altri prodotti precedentemente realizzati(esempio: sito web, app smartphone, ecc\dots).

I test con gli utenti hanno però evidenziato che temi di colori troppo scuri possono risultare pesanti alla vista e percepiti come difficili da leggere. Si è scelto quindi di utilizzare un tema di colori chiaro, con gli elementi più importanti in contrasto di colore in modo tale da indurre un focus immediato e guidare l'utente durante le operazioni.

\begin{figure}[!htbp]
\centering
\includegraphics[scale=0.30]{ux/bonificonero.png}
\caption{Tema iniziale nero per la sezione bonifico}
\end{figure}

\begin{figure}[!htbp]
\centering
\includegraphics[scale=0.30]{ux/bonificogrigio.png}
\caption{Tema chiaro per il bonifico nella sua versione finale, si noti il pulsante \emph{conferma} in risalto rispetto altri elementi della GUI}
\end{figure}

\begin{figure}[!htbp]
\centering
\includegraphics[scale=0.30]{ux/timeline3.png}
\caption{Tema verde per il background della timeline}
\end{figure}
\begin{figure}[!htbp]
\centering
\includegraphics[scale=0.30]{ux/timelinebianca.png}
\caption{Tema chiaro per il background della timeline e utilizzo del colore rosso e grigio per dare risalto ai saldi negativi e positivi dei movimenti}
\end{figure}

\subsubsection{Suggerimenti}

Dai test effettuati è emersa la necessita di aiutare l'utente attraverso suggerimenti (\emph{hint}) visivi posti in particolari punti delle interfacce. L'utilizzo volutamente moderato di suggerimenti ha aumentato e migliorato la \emph{discoverability}. 
\begin{figure}[!htbp]
\centering
\includegraphics[scale=0.30]{ux/timelinenohint.png}
\caption{L'utente premeva il pulsante \emph{menu} per cercare di tornare alla schemata dei conti}
\end{figure}
\begin{figure}[!htbp]
\centering
\includegraphics[scale=0.30]{ux/timelinehint.png}
\caption{Con una freccia che rappresentasse il significato di \emph{indietro} si è messo in risalto la relativa azione}
\end{figure}

\newpage
\subsubsection{Feedback}
Per indicare che, in risposta ad una azione dell'utente, è avvenuto un evento nel sistema  si è scelto di utilizzare feedback visivi e diversificati in base al tipo di azione intrapresa. 
Di seguito sono riportati alcuni dei feedback utilizzati all'interno dell'applicazione:

\begin{itemize}
 \item componenti che indicano il caricamento durante task di lunga durata (come ad esempio durante le chiamate http ai servizi, figura \ref{fig:loading})
 \item l'inserimento di animazioni per passare da un'interfacca ad un'altra dopo l'azione eseguita dall'utente (figura \ref{fig:animation})
 \item elementi grafici che cambiano il loro stato in risposta a eventi particolari (figura \ref{fig:copy})

\end{itemize}


\begin{figure}[htbp]
\centering
\includegraphics[scale=0.30]{ux/loading.png}
\caption{Esempi di loading view durante la chiamata ai servizi}
\label{fig:loading}
\end{figure}

\begin{figure}[htbp]
\centering
\includegraphics[scale=0.30]{ux/animazione.png}
\caption{Fotogramma preso durante un'animazione rappresentante il cambio di interfaccia e del relativo contesto (colori, immagini, contenuti)}
\label{fig:animation}
\end{figure}

\begin{figure}[htbp]
\centering
\includegraphics[scale=0.30]{ux/copia.png}
\caption{Dopo l'azione di copia il relativo pulsante cambia per qualche frazione di secondo il suo contenuto in \emph{copiato!}}
\label{fig:copy}
\end{figure}

\subsubsection{Componenti personalizzati}
L'utilizzo di componenti grafici personalizzati ha permesso di ottenere interfacce grafiche più ricche di funzioni e al tempo stesso intuitive e facili da usare. La figura \ref{fig:selettoreC} rappresenta un componente grafico che ha la funzione di poter selezionare attraverso un tap sui pulsanti o attraverso la gesture swipe un elemento in una lista di oggetti (in questo caso di conti); tale componente è stato realizzato come libreria esterna, e quindi riusabile in diversi contesti della stessa applicazione.
\begin{figure}[htbp]
\centering
\includegraphics[scale=0.50]{ux/selettore.png}

\caption{Il componente custom \emph{selettore conti}}
\label{fig:selettoreC}
\end{figure}

Un altro esempio di componente personalizzato sono i tasti di azione rapida accessibili effettuando la gesture \emph{swipe} sulle righe della \emph{Message box} interna all'applicazione.
In questo caso si è ricreato lo stesso design e si è mantenuta la stessa interazione che presenta l'applicazione \emph{Mail} nativa di iOS 7, permettendo così all'utente di riutilizzare le stesse abilità acquisite nell'utilizzo di un software originale Apple.
\begin{figure}[!htbp]
\centering
\includegraphics[scale=0.50]{ux/messageBox.png}

\caption{Dettaglio delle funzionalità della \emph{Message box}}
\label{fig:selettore}
\end{figure}
\subsubsection{Help}
Per permettere un facile e rapido utilizzo dell'applicazione a tutti i possibili utenti, che spaziano dall'utilizzatore abituale di dispositivi e applicazioni mobili a quelli che si avvicinano per la prima volta alle nuove tecnologie, si è deciso di utilizzare al primo avvio dell'app alcune schermate di aiuto (figura \ref{fig:help}) con la finalità di descrivere le principali  e le meno intuitive funzionalità del software.

Le fasi di test hanno evidenziato l'effettiva necessità di tale scelta che ha portato al miglioramento della discoverability, soprattutto nei primi utilizzi dell'app da parte dell'utente.

\begin{figure}[!htbp]
\centering
\includegraphics[scale=0.20]{ux/help.jpg}

\caption{Schermata di help per descrivere le funzionalità dell'interfaccia di riepilogo di conti e carte}
\label{fig:help}
\end{figure}


\subsubsection{Prestazioni}
Per rendere più gradevole la user experience si è deciso di eseguire il \emph{caching}\footnote{Eseguire il salvataggio in ram o in memoria secondaria dei dati di frequente utilizzo, in modo tale da renderli disponibili per accessi futuri.} di alcuni dei dati scaricati dalla rete (come ad esempio la lista dei prodotti, o dati relativi al gestore, ecc\dots). Questa scelta ha quindi permesso di ridurre i tempi di accesso ad alcuni servizi migliorando le prestazioni del software in generale e l'interazione dell'utente con lo stesso.  
\chapter{Casi d'suo}
I casi d'uso permettono di modellare il comportamento del sistema e descrivere i requisiti funzionali in linguaggio naturale. Grazie a questi modelli è quindi possibile rappresentare il comportamento esterno del sistema, visto come una black-box, dal punto di vista di un attore.

Oltre a una descrizione testuale (gli \emph{Scenari}) i casi d'uso possono esser rappresentati mediante diagrammi. In tali diagrammi troviamo le segunti entità:
 \begin{itemize}
  \item gli attori: essi rappresentano un soggetto o un'entità esterna che interagisce col sistema
  \item i casi d'uso: rappresentano un particolare scenario di interazione tra attore e sistema. Sono rappresentati graficamente da ellissi
  \item le associazioni: rappresentate da una linea mettono in comunicazione un attore con i caso d'uso
 \end{itemize}

 Nei prossimi paragrafi saranno mostrati alcuni dei casi d'uso modellati durante la fase di analisi.
 \section{Diagrammi dei casi d'uso}
 \subsection{Caso d'uso 1\_0 Ricarica telefonica}
	\begin{figure}[!htbp]
	  \centering
	  \includegraphics[scale=0.60]{casi_uso/ricarica.png}
	\end{figure}
 
 
 \section{Specifica dei casi d'uso}
 \subsection{Caso d'uso: Ricarica telefonica}

\begin{center}
     \begin{longtable}{{ | l | p{8cm} |}}
    \hline
    \textbf{Caso d'uso} UC\_1\_0 & Ricarica telefonica di un numero salvato in rubrica \\ \hline
    \textbf{Attore} & Correntista bancario  \\ \hline
    \textbf{Precondizioni} & L'utente è loggato al sistema e ha iniziato una sessione inserendo un codice OTP  \\ \hline
    \textbf{Postcondizioni di successo}  & Il sistema ha registrato l'operazione \\\hline
    \textbf{Postcondizioni di fallimento}   &  Il sistema non ha registrato l'operazione\\\hline
    \textbf{Scenario} &  \\\hline
    & \begin{enumerate}
       \item L'utente seleziona l'operazione di ricarica telefonica
       \item \texttt{Include} UC\_1\_1 \emph{Visualizza lista beneficiari}
       \item \texttt{Include} UC\_1\_2 \emph{Visualizza lista conti di addebito}
       \item \texttt{Include} UC\_1\_3 \emph{Visualizza lista operatori}
       \item \texttt{Include} UC\_1\_4 \emph{Verifica pin e OTP}
       
%        \item \label{item:sel1}Il sistema visualizza l'elenco dei beneficiari
%        \item L'utente seleziona un beneficiario 
%        \item \label{item:sel2}Il sistema visualizza l'elenco dei conti per l'addebbito
%        \item L'utente seleziona un conto
%        \item \label{item:sel3}\label{item:compagnia} Il sistema mostra l'elenco delle compagnie telefoniche
%        \item L'utente seleziona una compagnia telefonica
%        \item \label{item:sel4}Il sistema visualizza i tagli disponibili
%        \item L'utente seleziona un taglio di ricarica
%        \item \label{item:sel5}\label{item:beneficiario} Il sistema mostra il riepilogo dell'operazione e richiede conferma
%        \item L'utente conferma i dati
%        \item \label{item:credenziali}\label{item:sel6}Il sistema richiede Pin e OTP
%        \item L'utente immette i dati richiesti e conferma
%        \item \texttt{Include} \emph{Verifica credenziali}
       \item Il sistema mostra lo scontrino dell'operazione
      \end{enumerate}\\\hline
      \textbf{Scenari alternativi} &  \\\hline
     & \begin{itemize}
       \item  \ref{item:sel1}, \ref{item:sel2}, \ref{item:sel3}, \ref{item:sel4}, \ref{item:sel5}, \ref{item:sel6} L'utente annulla operazione
       \item Il caso d'uso termina
      \end{itemize}\\\hline
    \textbf{Scenari di errore} &  \\\hline
    & \begin{enumerate}
    \setcounter{enumi}{6}
       \item L'utente seleziona una compagnia telefonica errata
       \item Il sistema mostra messaggio di errore
       \item Il caso d'uso riprende dal punto \ref{item:compagnia}
      \end{enumerate}\\\hline
           & \begin{enumerate}
    \setcounter{enumi}{13}
       \item Il sistema non valida le credenziali
       \item Il sistema comunica tipo di errore
       \item Il caso d'uso riprende dal punto \ref{item:credenziali}
      \end{enumerate}\\\hline

     \end{longtable}
\end{center}

\begin{center}
     \begin{longtable}{{ | l | p{8cm} |}}
    \hline
    \textbf{Caso d'uso} UC\_1\_1 & Selezione beneficiario \\ \hline
    \textbf{Attore} & Correntista bancario  \\ \hline
    \textbf{Precondizioni} & L'utente ha iniziato una nuova ricarica telefonica \\ \hline
    \textbf{Postcondizioni di successo}  & Il sistema ha convalidato la scelta del beneficiario \\\hline
    \textbf{Postcondizioni di fallimento}   &  Il sistema non ha cambiato stato\\\hline
    \textbf{Scenario} &  \\\hline
    & \begin{enumerate}
       \item \label{item:beneficiario} Il sistema visualizza la lista dei beneficiari
       \item L'utente seleziona un beneficiario 
       \item Il sistema aggiorna l'interfaccia con la selezione effettuata 
      \end{enumerate}\\\hline
      \textbf{Scenari alternativi} &  \\\hline
    & \begin{enumerate}
    \setcounter{enumi}{1}
       \item L'utente seleziona la voce \emph{Beneficiario}
       \item Il caso d'uso riprende dal punto \ref{item:beneficiario}
      \end{enumerate}\\\hline
     & \begin{itemize}
       \item L'utente annulla l'operazione
       \item Il caso d'uso termina
      \end{itemize}\\\hline
    \textbf{Scenari di errore} &  \\\hline
    & \begin{enumerate}
    \setcounter{enumi}{2}
       \item Il sistema non valida le credenziali
       \item Il sistema comunica il tipo di errore
       \item Il caso d'uso riprende dal punto \ref{item:beneficiario}
      \end{enumerate}\\\hline

     \end{longtable}
\end{center}

\begin{center}
     \begin{longtable}{{ | l | p{8cm} |}}
    \hline
    \textbf{Caso d'uso} UC\_1\_2 & Selezione conto di addebito \\ \hline
    \textbf{Attore} & Correntista bancario  \\ \hline
    \textbf{Precondizioni} & L'utente ha iniziato una nuova ricarica telefonica \\ \hline
    \textbf{Postcondizioni di successo}  & Il sistema ha convalidato la scelta del conto \\\hline
    \textbf{Postcondizioni di fallimento}   &  Il sistema non ha cambiato stato\\\hline
    \textbf{Scenario} &  \\\hline
    & \begin{enumerate}
       \item \label{item:conto} Il sistema visualizza la lista dei conti disponibili
       \item L'utente seleziona un conto 
       \item Il sistema aggiorna l'interfaccia con la selezione effettuata 
      \end{enumerate}\\\hline
      \textbf{Scenari alternativi} &  \\\hline
    & \begin{enumerate}
    \setcounter{enumi}{1}
       \item L'utente seleziona la voce \emph{Conto di addebito}
       \item Il caso d'uso riprende dal punto \ref{item:conto}
      \end{enumerate}\\\hline
     & \begin{itemize}
       \item L'utente annulla l'operazione
       \item Il caso d'uso termina
      \end{itemize}\\\hline
     \end{longtable}
\end{center}

\begin{center}
     \begin{longtable}{{ | l | p{8cm} |}}
    \hline
    \textbf{Caso d'uso} UC\_1\_3 & Selezione operatore \\ \hline
    \textbf{Attore} & Correntista bancario  \\ \hline
    \textbf{Precondizioni} & L'utente ha iniziato una nuova ricarica telefonica \\ \hline
    \textbf{Postcondizioni di successo}  & Il sistema ha convalidato la scelta dell'operatore \\\hline
    \textbf{Postcondizioni di fallimento}   &  Il sistema non ha cambiato stato\\\hline
    \textbf{Scenario} &  \\\hline
    & \begin{enumerate}
       \item \label{item:compagnia}Il sistema visualizza la lista delle compagnie telefoniche
       \item L'utente seleziona una compagnia 
       \item \label{item:tagli} Il sistema mostra i tagli di ricarica disponibili
       \item L'utente seleziona un tagio di ricarica
       \item Il sistema aggiorna l'interfaccia con la selezione effettuata
      \end{enumerate}\\\hline
      \textbf{Scenari alternativi} &  \\\hline
    & \begin{enumerate}
    \setcounter{enumi}{1}
       \item L'utente seleziona la voce \emph{Importo}
       \item Il caso d'uso riprende dal punto \ref{item:tagli}
      \end{enumerate}\\\hline
     & \begin{itemize}
       \item L'utente annulla l'operazione
       \item Il caso d'uso termina
      \end{itemize}\\\hline
    \textbf{Scenari di errore} &  \\\hline
    & \begin{enumerate}
    \setcounter{enumi}{1}
       \item L'utente seleziona una compagnia telefonica errata
       \item Il sistema mostra messaggio di errore
       \item Il caso d'uso riprende dal punto \ref{item:compagnia}
      \end{enumerate}\\\hline

     \end{longtable}
\end{center}

\begin{center}
     \begin{longtable}{{ | l | p{8cm} |}}
    \hline
    \textbf{Caso d'uso} UC\_1\_4 & Verifica Pin e OTP \\ \hline
    \textbf{Attore} & Correntista bancario  \\ \hline
    \textbf{Precondizioni} & L'utente ha iniziato una nuova ricarica telefonica e ha immesso tutti i dati richiesti\\ \hline
    \textbf{Postcondizioni di successo}  & Il sistema ha convalidato l'operazione e ha aggiornato il suo stato \\\hline
    \textbf{Postcondizioni di fallimento}   &  Il sistema non ha convalidato l'operazione e ha mostrato un messaggio di errore\\\hline
    \textbf{Scenario} &  \\\hline
    & \begin{enumerate}
       \item  Il sistema mostra il riepilogo dell'operazione e richiede conferma
       \item L'utente conferma i dati
       \item \label{item:credenziali}Il sistema richiede Pin e OTP
       \item L'utente immette i dati richiesti e conferma
       \item Il sistema verifica le credenziali
       \item Il sistema mostra lo scontrino dell'operazione effettuata
     \end{enumerate}\\\hline
      \textbf{Scenari alternativi} &  \\\hline
     & \begin{itemize}
       \item L'utente annulla l'operazione
       \item Il caso d'uso termina
      \end{itemize}\\\hline
     & \begin{enumerate}
      \setcounter{enumi}{4}
       \item Il sistema non valida le credenziali
       \item Il sistema visualizza tipo il di errore
       \item Il caso d'uso riprende dal punto \ref{item:credenziali}
      \end{enumerate}\\\hline
     \end{longtable}
\end{center}

 \subsection{Caso d'uso 1\_0 Ricarica telefonica}
	\begin{figure}[!htbp]
	  \centering
	  \includegraphics[scale=0.60]{casi_uso/conti.png}
	\end{figure}
\subsection{Caso d'uso: Consultare lista conti e carte}

\begin{center}
     \begin{longtable}{{ | l | p{8cm} |}}
    \hline
    \textbf{Caso d'uso} & UC\_2\_0  Visualizza lista conti e carte \\ \hline
    \textbf{Attore} & Correntista bancario  \\ \hline
    \textbf{Precondizioni} & L'utente ha effettuato un login di primo o secondo livello \\ \hline
    \textbf{Postcondizioni di successo}  & Il sistema rimane nel suo stato o aggiorna la lista dei conti in memoria \\\hline
    \textbf{Postcondizioni di fallimento}   &  Il sistema rimane nel suo stato\\\hline
    \textbf{Scenario} &  \\\hline
    & \begin{enumerate}
       \item L'utente seleziona l'operazione di visualizzazione lista conti e carte
       \item \label{item:verificalista}Il sistema verifica che l'elenco dei prodotti non è vuoto
       \item Il sistema mostra la lista dei prodotti
      \end{enumerate}\\\hline
    \textbf{Scenario alternativo} &  \\\hline
    & \begin{enumerate}
    \setcounter{enumi}{1}
       \item Il sistema ha verificato che la lista dei prodotti è vuota
       \item Il sistema notifica l'evento con un opportuno messaggio
	\item Il caso d'uso termina
       \end{enumerate}\\\hline
    \textbf{Scenario di errore} &  \\\hline
    & \begin{enumerate}
    \setcounter{enumi}{1}
       \item Il sistema riceve un errore
       \item Il sistema notifica l'evento con un messaggio di errore e la possibilità di riprovare l'azione
       \item L'utente seleziona \emph{riprova}
       \item Il caso d'uso riprende dal punto \ref{item:verificalista}
       \end{enumerate}\\\hline
     \end{longtable}
\end{center}

\begin{center}
     \begin{longtable}{{ | l | p{8cm} |}}
    \hline
    \textbf{Caso d'uso} & UC\_2\_1  Visualizza lista movimenti \\ \hline
    \textbf{Attore} & Correntista bancario  \\ \hline
    \textbf{Precondizioni} & L'utente ha effettuato un login di secondo livello \\ \hline
    \textbf{Postcondizioni di successo}  & Il sistema rimane nel suo stato \\\hline
    \textbf{Postcondizioni di fallimento}   &  Il sistema rimane nel suo stato\\\hline
    \textbf{Scenario} &  \\\hline
    & \begin{enumerate}
       \item \texttt{Include} UC\_2\_0 \emph{Visualizza lista conti e carte}
       \item L'utente seleziona un conto o una carta
       \item \label{item:verificamovimenti}Il sistema verifica che la lista dei movimenti non è vuoto
       \item Il sistema mostra la lista dei prodotti
      \end{enumerate}\\\hline
    \textbf{Scenario alternativo} &  \\\hline
    & \begin{enumerate}
    \setcounter{enumi}{1}
       \item Il sistema ha verificato che la lista dei movimenti è vuota
       \item Il sistema notifica l'evento con un opportuno messaggio
	\item \texttt{Extend} \emph{Filtra movimenti per data}
       \end{enumerate}\\\hline
    \textbf{Scenario di errore} &  \\\hline
    & \begin{enumerate}
    \setcounter{enumi}{1}
       \item Il sistema riceve un errore
       \item Il sistema notifica l'evento con un messaggio di errore e la possibilità di riprovare l'azione
       \item L'utente seleziona \emph{riprova}
       \item Il caso d'uso riprende dal punto \ref{item:verificamovimenti}
       \end{enumerate}\\\hline
     \end{longtable}
\end{center}

\chapter{Architettura software}


\section{Architettura generale}

Il progetto è stato realizzato seguendo un'architettura software client-server (figura \ref{fig:arch}). In questa architettura i componenti principali sono due: il client rappresentato dall'applicazione iPad oggetto di questo documento e il lato server scomposto in \emph{middleware} e nei sistemi messi a disposizione dal cliente (l'istituto bancario).

Il middleware opera da intermediario tra i servizi bancari e l'applicazione ed  ha il compito di ricevere le richieste dal client, di sottomettere a sua volta queste richieste ai servizi lato banca e di ritornare i dati al client opportunamente codificati.

Le richieste verso il middleware (ospitante un webserver \emph{Apache}) vengono eseguite seguendo un'architettura \emph{RESTFUL}  che prevede richieste \emph{HTTP} (POST, GET, PUT, DELETE). Le risposte fornite dal middleware verso il client sono costruite in formato \emph{JSON}.


\begin{figure}[!htbp]
\centering
\includegraphics[scale=0.60]{architettura/architect.png}
\caption{Architettura client-server}
\label{fig:arch}
\end{figure}

\section{Architettura dell' applicazione}

L'applicazione è stata progettata usando il design pattern \emph{MVC} (Model-View-Controller, figura \ref{fig:mvc}). In tale pattern ogni oggetto dell'applicazione assolve uno dei seguenti ruoli:
\begin{itemize}
 \item Model: gli oggetti di tipo model hanno il compito di incapsulare i dati specifici di un'applicazione e di definire le logiche per modificare e processare tali dati
 \item View: sono gli oggetti dell'applicazione visibili all'utente. Tali oggetti hanno il compito di visualizzare i dati dell'applicazione e di permetterne l'interazione (esempio la modifica)
 \item Controller: è un oggetto che opera da intermediario tra la view e uno o più model dell'applicazione. Tali oggetti hanno quindi la funzione di comunicare i cambiamenti tra le view e i model, e viceversa.
\end{itemize}

L'utilizzo del pattern MVC permette una migliore estensione del codice, di costruire componenti riusabili e migliorare la definizione delle interfacce.

\begin{figure}[!htbp]
\centering
\includegraphics[scale=0.70]{architettura/mvc.png}
\caption{Schema del pattern MVC}
\label{fig:mvc}
\end{figure}

Nei prossimi paragrafi verranno descritte le classi principali del progetto.

\section{Diagramma delle classi}

\subsection{Framework e librerie esterne}
Oltre agli strumenti messi a disposizione dalla Apple sono stati utilizzati framework esterni (open source) che hanno permesso di ridurre i tempi di sviluppo, di seguito due esempi:
\begin{itemize}
 \item CorePlot: è un framework per il disegno di grafici in applicazioni iOS, fornsisce strumenti per il disgeno e la visualizzazione dei dati in 2D
 \item JNKeychain: wrapper sulle funzioni del \emph{KeyChain}, permette il salvataggio e il recupero dei dati in maniera sicura dei dati confindenziali
\end{itemize}


\subsection{Business Logic}
Di seguito sono rappresentate una parte delle classi della \emph{business logic}.
\begin{figure}[!htbp]
\centering
\includegraphics[scale=0.60]{architettura/businessLogicClass.png}
\caption{Classi della business logic}
\label{fig:businessLogic}
\end{figure}

\subsection{Connessione ai servizi}
\label{parag:networking}
La parte relativa alla gestione delle chiamate ai servizi è stata realizzata implementando classi di tipo \emph{singleton}. Tali classi estendono la libreria di terze parti \texttt{AFNetworking} che opera da wrapper sulle tecnologie di comunicazione messe a disposizione dall'SDK iOS (Foundation URL Loading System). 

Utilizzare una libreria come AFNetworking ha permesso di ridurre i tempi di sviluppo e di sfruttare le potenzialità offerte dalle sue API come: la serializzazione e deserializzazione dei dati, la possibilità di effettuare richieste HTTP asincrone, una gestione efficiente delle code, la gestione di sessioni, ecc\dots.  

\begin{figure}[!htbp]
\centering
\includegraphics[scale=0.70]{architettura/networkingClass.png}
\caption{Diagramma delle classi per le connessioni di rete}
\end{figure}

\subsubsection{Autorizzazione}

La classe singleton \texttt{AuthHTTPClient} ha il compito di effettuare il login ai servizi e di ottenere le credenziali di accesso alle risorse protette. Per l'autorizzazione e  l’accesso alle risorse offerte dai servizi bancari è stato utilizzato il protocollo open \emph{OAut 2.0} (Open Authorization).

Nella classica autenticazione client-server, il client ottiene l’accesso alle risorse protette, ospitate sul server, richiedendo l’autenticazione ad esso attraverso la sottomissione delle credenziali del proprietario delle risorse. Quindi per fornire ad ogni applicazione l’accesso alle risorse protette, il proprietario di tale risorse deve condividere le sue credenziali con queste applicazioni. Questo può portare a diversi problemi di sicurezza e limitazioni (argomenti non trattati in questo documento).
 
Nel protocollo OAuth viene fatta una separazione tra il client e il proprietario delle risorse aggiungendo un ulteriore layer che si occupa dell’autorizzazione delle applicazioni di terze parti.
Invece di usare le credenziali del proprietario delle risorse il client utilizza un \emph{access token} (una stringa che indica un preciso scope, tempo di vita, ecc\dots).
Questo access token è fornito da un \emph{Authorization Server} previo permesso del proprietario delle risorse.

Le entità coinvolte in questo protocollo sono:



\begin{itemize}
  \item Resource server: il sistema in cui sono memorizzate le risorse
  \item Resource owner: il proprietario delle risorse, colui che possiede le credenziali sul Resource server
  \item Protected resource: dati controllati dal Resource server
  \item Client: un’applicazione che usa OAuth per accedere alle risorse ospitate sul Resource server, sotto la delega del Resource owner
\end{itemize}



Le figure \ref{fig:oauth} e \ref{fig:refresh} mostrano lo scambio dei dati tra un Client, un Authorization Server e un Resource Server nel caso in cui viene richiesto per la prima volta l'autenticazione e nel caso in cui viene generato un nuovo token a partire da un \emph{Refresh Token}.

\begin{figure}[!htbp]
\centering
\includegraphics[scale=0.50]{architettura/token.png}
\caption{Nel diagramma è mostrato il flusso di richiesta di un token per l'accesso alle risorse. Il client richiede al Resource Owner il permesso per l'autenticazione. Il Resource Owner invia al Client il permesso rappresentato come un codice che segue le specifiche del protocollo. Il Client richiede all'Authorization Serve un access token passandogli il codice di autorizzazione. L'Authorization Server autentica il client e ritorna il token. Il Client utilizzerà tale token per accedere alle risorse protette memorizzate nel Resource Server. }
\label{fig:oauth}
\end{figure}


\begin{figure}[!htbp]
\centering
\includegraphics[scale=0.40]{architettura/refreshToken.png}
\caption{Il diagramma mostra il caso in cui la richiesta ad una risorsa protetta avvenga passando al Resource Server un token valido e il caso in cui il token corrente sia non valido. In quest'ultimo caso viene generato un nuovo token a partire dal \emph{refresh token}. }
\label{fig:refresh}
\end{figure}

% i servizi sono separati da quelli di autenticazione perchè gli  standard oauth richiedono la separazione i sottodimini diversi.
\newpage
\subsection{Caching dei dati}
Per migliorare le prestazioni e limitare il traffico di rete generato durante l'utilizzo dell'applicazione è stato implementato un meccanismo di caching dei dati che consente di salvare in memoria primaria i dati necessari all'applicazione durante una sessione di utilizzo. 

\begin{figure}[!htbp]
\centering
\includegraphics[scale=0.70]{architettura/cacheClass.png}
\caption{Diagramma delle classi per il caching dei dati}
\end{figure}

I controller possono richiamare l'oggetto singleton \texttt{SharedDataHandler} e attraverso esso salvare in variabili temporanee i dati che non necessitano di essere riscaricati dalla rete durante una sessione .

\subsection{Autenticazione e sicurezza dei dati}
L'applicazione prevede un sistema di autenticazione ai servizi basata su livelli. In particolare sono previsti tre livelli di autenticazione:
\begin{itemize}
 \item autenticazione di livello zero: è il livello associato a un utente che non ha ancora effettuato il login
 \item autenticazione di primo livello: è il livello associato a un utente che ha precedentemente effettuato il login, ha scelto di ricordare le credenziali ed ha momentaneamente sospeso l'utilizzo dell'applicazione. Questo livello permette all'utente di accedere a un gruppo ristretto di funzionalità nel momento in cui riprenderà l'utilizzo dell'applicazione
 \item autenticazione di secondo livello: è il livello associato a un utente che ha precedentemente effettuato il login e che permette l'accesso a tutte le funzionalità previste dall'applicazione. Questo livello permane fino a quando l'utente non richiede esplicitamente il logout o sospende l'applicazione
\end{itemize}

\begin{figure}[!htbp]
\centering
\includegraphics[scale=0.70]{architettura/stati.png}
\caption{Diagramma degli stati di autenticazione}
\end{figure}


L'autenticazione ai servizi è a carico delle classi descritte nel paragrafo \ref{parag:networking}. Questa operazione restituisce in output i dati confidenziali dell'utente che saranno gestiti dalle classi adibite alla sicurezza.

A tal fine è stato implementato un oggetto singleton, il \texttt{SecurityManager}, che ha la responsabilità di stabilire il livello di autenticazione in risposta agli eventi che occorrono all'interno dell'applicazione e di gestire il salvataggio e il recupero dei dati confidenziali dell'utente (come le informazioni sul token, sul numero utente, ecc\dots). Tale classe utilizza la libreria di terze parti \texttt{JNKeychain} che estende le funzioni messe a disposizione dell'SDK iOS, in particolare quelle relative al \emph{Keychain}. 

\begin{figure}[!htbp]
\centering
\includegraphics[scale=0.70]{architettura/securityClass.png}
\caption{Diagramma delle classi relative alla sicurezza}
\end{figure}
\newpage
\subsection{Controller principali}
Di seguito è riportato il diagramma delle classi rappresentanti i controller principali creati durante la progettazione e lo sviluppo. Per semplicità di esposizione sono stati rappresentati solo i nomi dei controller e le relazioni tra essi.
\newline

A titolo esemplificativo nel diagramma si può notare che il \texttt{BaseViewController} è il controller principale dal quale gli altri controller ereditano le funzionalità e proprietà generiche. Tale controller si occupa della creazione e visualizzazione degli  elementi grafici comuni a tutte le interfacce (ad esempio la navigation bar personalizzata) e delle logiche necessarie ad ogni controller che lo specializza.

In particolare uno dei compiti principali di questo controller è quello di creare il controller che gestisce l'interfaccia e la logica per il login ai servizi (\texttt{LoginViewController}) e di implementare i metodi nel  protocollo \texttt{LoginViewControllerProtocol} dichiarato dallo stesso LoginViewController. Sarà quindi compito del LoginViewController  lanciare il processo di login e comunicare al BaseViewController l'esito dell'operazione attraverso il protocollo descritto precedentemente; così facendo il BaseViewController sarà in grado di effettuare l'aggiornamento dello stato dell'interfaccia.


 \newpage
% \begin{landscape}
\begin{figure}[!htbp]
\centering
\includegraphics[scale=0.65]{architettura/controllersNew.png}
\caption{Diagramma delle classi controller}
\end{figure}
% \end{landscape}



\section{Diagramma di sequenza}

Di seguito è riportato un esempio di diagramma di sequenza relativo allo scenario in cui un utente richieda la lista dei conti e delle carte a lui registrate; in tale diagramma saranno evidenziati le sequenze di messaggi scambiati tra le entità che entrano in gioco durante il processo di recupero e visualizzazione delle informazioni. 
 \begin{landscape}
\begin{figure}[!htbp]
\centering
\includegraphics[scale=0.65]{architettura/sequence.png}
\caption{Diagramma si sequenza per la visualizzazione della lista conti e carte}
\end{figure}
\end{landscape}
\chapter{Dettagli implementativi}

In questo capitolo verranno descritti i pattern principali usati durante la progettazione e lo sviluppo del softaware e le motivazioni che hanno portato a tali scelte.

\section{Custom container controller}
L'\emph{SDK} iOS mette a disposizione diversi container controller
\chapter{Sviluppi futuri}

\section{Evoluzioni}

Il progetto prevede una futura evoluzione grazie alla realizzazione di nuove funzionalità che permettano all'utente finale di sfruttare servizi maggiormente personalizzati e più vicini alle sue necessità. L'insieme delle nuove funzionalità prese in considerazione e analizzate prevedono:

\begin{itemize}
 \item Bubble interattive e mobili: presentare all'utente il riepilogo dei propri conti e delle proprie carte sotto forma di bolle grafiche spostabili attraverso particolari \emph{gesture} sullo schermo. Oltre ad una pura funzione di intrattenimento questo meccanismo può essere sfruttato per poter decidere quali prodotti visualizzare e quali no semplicemente spostando le bubble al di fuori dello schermo, senza necessariamente utilizzare l'apposita funzione di filtro presente nell'applicazione.
 
 Per abbattere i tempi di sviluppo sono state analizzate e testate librerie di terze parti Open Source realizzate in Javascript che permettono una rapida integrazione attraverso il componente  \texttt{UIWebView}  nativo di iOS. Dai test è risultato che l'unione di queste due differenti tecnologie non porta a problemi di performance e di interazione rendendo quindi ideale tale scelta implementativa.
 
 \item Trova filiale: prevede l'utilizzo di servizi di geolocalizzazione delle filiali bancarie e di visualizzazione delle loro posizioni all'interno di una mappa. L'utilizzo di tali funzioni, integrate direttamente all'interno dell'applicazione, permetteranno all'utente di poter ricevere in poco tempo e con alta precisione informazioni relative alle filiali bancarie più vicine ad essi e di ricevere informazioni di dettaglio come ad esempio gli orari di apertura, i servizi offerti, ecc\dots
 
 \item Personalizzazione dei contenuti social e dei contenuti informativi: prevede l'utilizzo di meccasismi di analisi delle abitudini dell'utente al fine di poter capire quali sono i canali social e le notizie di maggior interesse per esso. Queste analisi permetteranno in futuro di poter presentare contenuti social e centenuti informativi specializzati, di qualità e differenziati in base al singolo utente
 
 
\end{itemize}


%  \chapter{Style features of \textsf{sapthesis}}
% 
% In this chapter I will discuss my stylistic choices of \textsf{sapthesis}.
% I will show the page layout geometry and I will describe the page style.
% 
% \section{Page layout}
% 
% The page is fixed at the dimensions of an A4 paper, therefore you have to print your thesis on A4 paper to obtain the best results. The font dimension is fixed at 11\, pt. The text column and the margins are chosen to fill to the best an A4 paper while keeping a reasonable line length (396\, pt) for a good readability. The text height and the text width are in golden ratio (\textasciitilde 1.6180) as well as the outer and inner margins in a two-side document after binding margin removal. Also the top margin (excluding the header) and bottom margin are in the golden ratio. In Fig.~\ref{layout} a sketch of the \textsf{sapthesis} page layout is shown.

% \begin{figure}[h]
% \centering
% \setlength{\unitlength}{0.27mm}
% \begin{picture}(420,297)(-210,0)
% \polyline(-210,0)(210,0)(210,297)(-210,297)(-210,0)
% \Line(0,0)(0,297)
% \put(27.05,37.4){\polygon(0,0)(139.2,0)(139.2,223.8)(0,223.8)}
% \put(-27.05,37.4){\polygon(0,0)(-139.2,0)(-139.2,223.8)(0,223.8)}
% \put(27.05,268.16){\polygon(0,0)(139.2,0)(139.2,4.22)(0,4.22)}
% \put(-27.05,268.16){\polygon(0,0)(-139.2,0)(-139.2,4.22)(0,4.22)}
% \end{picture}
% \caption{Page layout scheme of \textsf{sapthesis class} using a zero binding margin.}
% \label{layout}
% \end{figure}


% \section{Page style}
% 
% The captions have a smaller font respect to the text and the label is in boldface. The appearance of the margin notes has been improved.
% They have the same font dimension of the footnotes and are typed in italics.
% Moreover I defined a new command to typeset margin note aligned to the left on the right page and vice versa on the left page.
% Notice that if a binding margin greater than 1.5\, cm is used, the dimensions of the margin notes become too small and very ugly.
% Do not use them in this case.
% 
% The mathematical objects, figures and tables are numbered within the chapters (e.g. 1.1, 1.2,\ldots for the first chapter, 2.1, 2.2 for the second one and so on\ldots). See for example the number of this simple equation
% \begin{equation}
% x_{1,2}=\frac{-b\pm\sqrt{b^2-4ac}}{2a}
% \end{equation}
% 
% 
% The title page is automatically composed when the \texttt{\bs maketitle} command is given.
% The parameters needed for the title page, author, title, etc\ldots , are supplied by dedicated commands explained in the next section.
% Two copies of the university logo in \texttt{pdf} format, one for color printing and the other one for black and white printing, are supplied in the \textsf{sapthesis} package. They are shown in Fig.~\ref{fig:largenenough}.
% 
% \begin{figure}
% \centering
% \includegraphics[width=0.7\textwidth]{sapienza-MLred-pos}\\[3ex]
% \includegraphics[width=0.7\textwidth]{sapienza-MLblack-pos}
% \caption{Logo of the Sapienza -- University of Rome.}
% \label{fig:largenenough}
% \end{figure}
% 
% 
% 
% \section{About figures and tables}
% 
% As regards the image formats, please use vector images as much as possible! Use jpg images only for photographs! pdf\LaTeX\ supports the pdf, jpg and png formats.
% 
% A very simple table is show in Tab.~\ref{tab:letters}. Remember to typeset
% always the table caption above the table. Do not use vertical lines.
% 
% \begin{table}
% \caption{This is a simple table.}
% \label{tab:letters}
% \centering
% \begin{tabular}{lcc}
% \toprule
% Letter & Test & Test \\
% \midrule
% A & C & E \\
% B & D & F \\
% \bottomrule
% \end{tabular}
% \end{table}
% 
% 
% \section{A section}
% 
% In this manual you can skip the gray text because it is just dummy text.
% 
% \textcolor{gray}{\lipsum[1-10]}
% 
% 
% 
% \section{Another section}
% 
% In this manual you can skip the gray text because it is just dummy text.
% 
% \textcolor{gray}{\lipsum}
% 
% 
% \appendix
% \chapter{Special commands provided by \textsf{sapthesis}}
% 
% \textsf{Sapthesis} provides some special commands, particularly useful for scientific works. You can use for example the roman shape, instead of the italic, for the imaginary unit (\texttt{\bs iu}) and Napier's number (\texttt{\bs eu}):
% \begin{equation}
% \eu^{\iu\pi}+1=0
% \end{equation}
% 
% There are also two commands to speed up the writing of derivatives. In the following example we have used the commands \texttt{\bs der} and \texttt{\bs pder}):
% \begin{equation}
% \der{f}{x} \qquad \pder[2]{f}{y}
% \end{equation}
% 
% 
% \textsf{Sapthesis} provides also 4 commands to improve the writing of subscripts, \texttt{\bs rb} and \texttt{\bs tb}, and superscripts, \texttt{\bs rp} and \texttt{\bs tp}. Two of these commands, \texttt{\bs rb} and \texttt{\bs rp}, can be used both in text and in math mode and compose their argument in roman. The other two, \texttt{\bs tb} and \texttt{\bs tp}, can be used only in text mode and compose their argument as are. Here it is an usage example of \texttt{\bs rb} and \texttt{\bs rp}:
% \[
% a_b \neq a\rb{b}\qquad a^b \neq a\rp{b}
% \]
% And here it is an usage example of \texttt{\bs tb}: \emph{Cu\tb{It} indicates copper bought in Italy}. And a usage example of \texttt{\bs ts}: \emph{Cher G\tp{le} Napol\'eon}.
% 
% 
% Then several commands for the correct typesetting of unit of measurements are provided. For example the command \texttt{\bs un} typesets its argument in roman and leaves a thin space between the number and the unit: $25\un{m}$, $3.5\un{m/s}$. Other commands are: (\texttt{\bs g}) 45\g, (\texttt{\bs C}) 30\,\C, (\texttt{\bs A}) 12\,\A, (\texttt{\bs micro}) 40\,\micro m, (\texttt{\bs ohm}) 27\,\ohm. 
% 
% We have also \texttt{\bs x} as abbreviation of \texttt{\bs times}: \$7 \bs x 10\^{}5\$ gives $7 \x 10^5$. Then \texttt{\bs di} is the differential symbol which automatically insert the correct spacing.
% \[
% \int x \di x
% \]
% 
% Finally we have defined the color \textsf{sapred} which is the official color
% of Sapienza -- University of Rome. It is defined as RGB(130,36,51). \textcolor{sapred}{This text is written with the color \textsf{sapred}.}
% 
% In the following dummy text you can observe the usage of \texttt{\bs mnote} command which typesets fancy margin notes.
% 
% \textcolor{gray}{\lipsum}
% \marginpar{This is a fancy margin note!}
% \textcolor{gray}{\lipsum}

\backmatter
% bibliography
%\cleardoublepage
%\phantomsection
%\bibliographystyle{sapthesis} % BibTeX style
%\bibliography{bibliography} % BibTeX database without .bib extension

\begin{thebibliography}{}
\bibitem{rif2} Apple, \emph{iOS Secure Coding Guide}.\newline \url{https://developer.apple.com/library/mac/documentation/security/conceptual/SecureCodingGuide/Introduction.html}
\bibitem{rif2} Apple, \emph{iOS Human Interface Guidelines}.\newline \url{https://developer.apple.com/library/ios/documentation/userexperience/conceptual/MobileHIG/index.html}
\bibitem{rif5} Alan Dix, Janet Finlay, Gregory D. Abowd, Russell Beale, \emph{Interazione uomo-macchina}. 2004.
\bibitem{rif4} Apple, \emph{iOS Application Programming Guide}. \newline \url{http://developer.apple.com/library/ios/#documentation/iPhone/Conceptual/iPhoneOSProgrammingGuide/Introduction/Introduction.html}
\bibitem{rif5} Ian Sommerville, \emph{Ingegneria del software}. 2007.
\bibitem{rif3} Apple, \emph{API's reference documents}.\newline \url{http://developer.apple.com/library/ios/navigation/#section=Resource%20Types&topic=Reference}
\bibitem{rif67}\emph{Manifesto for Agile Software Development}. \url{ http://agilemanifesto.org/}
\bibitem{rif47}\emph{ The Scrum Primer}. \url{http://www.scrumprimer.org/}
\bibitem{rif97} \emph{Scrum Alliance Why Scrum?}.  \url{http://www.scrumalliance.org/why-scrum}

\end{thebibliography}


\end{document}
