\chapter{Sviluppi futuri}

\section{Evoluzioni}

Il progetto prevede una futura evoluzione grazie alla realizzazione di nuove funzionalità che permettano all'utente finale di sfruttare servizi maggiormente personalizzati e più vicini alle sue necessità. L'insieme delle nuove funzionalità prese in considerazione e analizzate prevedono:

\begin{itemize}
 \item Bubble interattive e mobili: presentare all'utente il riepilogo dei propri conti e delle proprie carte sotto forma di bolle grafiche spostabili attraverso particolari \emph{gesture} sullo schermo. Oltre ad una pura funzione di intrattenimento questo meccanismo può essere sfruttato per poter decidere quali prodotti visualizzare e quali no semplicemente spostando le bubble al di fuori dello schermo, senza necessariamente utilizzare l'apposita funzione di filtro presente nell'applicazione.
 
 Per abbattere i tempi di sviluppo sono state analizzate e testate librerie di terze parti Open Source realizzate in Javascript che permettono una rapida integrazione attraverso il componente  \texttt{UIWebView}  nativo di iOS. Dai test è risultato che l'unione di queste due differenti tecnologie non porta a problemi di performance e di interazione rendendo quindi ideale tale scelta implementativa.
 
 \item Trova filiale: prevede l'utilizzo di servizi di geolocalizzazione delle filiali bancarie e di visualizzazione delle loro posizioni all'interno di una mappa. L'utilizzo di tali funzioni, integrate direttamente all'interno dell'applicazione, permetteranno all'utente di poter ricevere in poco tempo e con alta precisione informazioni relative alle filiali bancarie più vicine ad essi e di ricevere informazioni di dettaglio come ad esempio gli orari di apertura, i servizi offerti, ecc\dots
 
 \item Personalizzazione dei contenuti social e dei contenuti informativi: prevede l'utilizzo di meccasismi di analisi delle abitudini dell'utente al fine di poter capire quali sono i canali social e le notizie di maggior interesse per esso. Queste analisi permetteranno in futuro di poter presentare contenuti social e centenuti informativi specializzati, di qualità e differenziati in base al singolo utente
 
 
\end{itemize}
