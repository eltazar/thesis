\chapter{Metodologie usate e raffinamenti successivi}

\section{Metodologia Agile}

L'intero ciclo di vita del software è stato gestito adottando una metodologia \emph{Agile}.

I metodi Agile sono tali da coinvolgere il più possibile il committente, dando quindi vita a un processo di tipo adattativo: cioè che si adatta alle esigenze del cliente, che possono cambiare durante lo sviluppo.
L'Agile è un processo costituito da finestre di tempo limitate (2-4 settimane) chiamate iterazioni, le quali sono a loro volta scomposte nelle fasi di progettazione, di sviluppo e di test.

\begin{figure}[!htbp]
\centering
\includegraphics[scale=0.5]{immagini/agile.png}
\caption{Iterazioni e fasi della metodologia Agile}
\end{figure}

Il progetto è quindi suddiviso in singoli componenti indipendenti dalle funzionalità così da poterne analizzare e valutare i costi e i tempi. Ogni iterazione conterrà quindi tutto ciò che è indispensabile per rilasciare un piccolo incremento nelle funzionalità del software e sono tali da essere soggette a modifiche al fine di migliorare l'architettura finale del software.

\section{Requisiti funzionali finali}

Una progettazione iterativa e focalizzata sull'utente ha permesso quindi di ampliare e raffinare i requisiti fino ad ottenere  un risultato sempre più vicino alle richieste e alle necessità del committente.  

Per brevità di seguito sono riportati solo alcuni dei requisiti finali del progetto (che vanno ad aggiungersi o a integrare quelli della tabella \ref{tab:requisiti_iniziali}), per comodità raccolti in categorie.

\begin{center}
    \captionof{table}{Requisiti funzionali finali}

    \begin{longtable}{p{6cm}|p{8cm}}

    \toprule
    \multicolumn{1}{c}{\textbf{Requisito funzionale}} &
    \textbf{Descrizione}\\

    \midrule
    Accesso ai servizi & \begin{itemize}
                          \item Permettere l'accesso ai servizi attraverso credenziali
                          \item Permette l'accesso veloce a un sottoinsieme di funzioni salvando le credenziali
                          \item Permettere recupero Codice Cliente
                         \end{itemize}\\\\

    Riepilogo movimenti & \begin{itemize}
                           \item Rappresentare i movimenti di un conto attraverso una timeline
                           \item Mostrare i movimenti di un conto in una lista
                           \item Visualizzare una scheda di dettaglio per un movimento selezionato
                          \end{itemize}\\\\
    Filtro movimenti & Offrire la possibilità di filtrare i movimenti per data e per tipo (entrate, uscite, tutti)\\\\
    Riepilogo conti e carte & \begin{itemize}
				\item Visualizzare i prodotti di un utente attraverso bubble contenenti informazioni di riepilogo
				\item Visualizzare i prodotti di un utente attraverso una lista
				\item Permettere la selezione di un determinato prodotto
                              \end{itemize}\\\\
    Filtro conti e carte & Permettere la visualizzazione o meno di certi prodotti mediante un filtro\\\\
    Storico saldi & Visualizzare in un grafico a bolle lo storico dei saldi di un prodotto per mesi o settimane \\\\
    Disporre operazioni bancarie & \begin{itemize}
				      \item Fornire un form per la compilazione di dispositive 
				      \item Permettere la riproposizione di dispositive già effettuate
				    \end{itemize}\\\\
    Riepilogo dispositive effettuate & \begin{itemize}
                                        \item Visualizzare le ultime $n$ dispositive in una lista ordinata
                                        \item Visualizzare le ultime $n$ dispositive in un grafico a bolle
                                        \item Permettere il salvataggio delle dispositive effettuate in una lista di preferiti
                                       \end{itemize}\\\\
    Preferiti & Permettere la visualizzazione e la modifica di una lista di dispositive scelte come preferite\\\\\
    Rubrica & Permettere l'accesso in lettura e scrittura alla rubrica del dispositivo per poter salvare o recuperare numeri telefonici\\\\
    
    Help e gestore & \begin{itemize}
                      \item Offrire una sezione interna al software in cui l'utente può richiedere appuntamenti con un consulente
                      \item Mettere a disposizione una sezione di help e numeri utili
                     \end{itemize}\\\\

    Integrazione social network & \begin{itemize}
                                   \item Visualizzare i contenuti dei canali social offerti dal committente
                                   \item Offrire la possibilità di condividere i contenuti sui diversi social network
				   \item Integrare un player video per la visualizzazione di filmati relativi alle attività del committente
                                  \end{itemize}\\
     Browsing interno & Permettere la navigazione tra contenuti web attraverso un browser interno all'applicazione\\\\
     Notifiche Push & Permettere la ricezione di notifiche push\\\\
     Messaggistica		 & \begin{itemize}
                                   \item Visualizzare i messaggi ricevuti dai servizi
                                   \item Permettere la gestione dei messagi (esempio: cancella, segna come letto, ecc\dots)
                                  \end{itemize}\\
    \bottomrule

    \end{longtable}
\end{center}

Sono inoltre elencati alcuni dei requisiti non funzionali:

\begin{center}
    
    \captionof{table}{Requisiti non funzionali}

    \begin{tabular}{p{6cm}|p{8cm}}

    \toprule
    \multicolumn{1}{c}{\textbf{Requisito non funzionale}} &
    \textbf{Descrizione}\\

    \midrule
    Comunicazione sicura & \begin{itemize}
                            \item Garantire la comunicazione sicura tra client e server
                            \item Garantire che i dati salvati all'interno del device siano protetti
                           \end{itemize}\\\\
    Ottimizzazione flusso dati & Limitare il più possibile il consumo dei dati\\\\
    Efficienza & Garantire uso efficiente delle risorse offerte dal device (cpu, batteria, ecc\dots)\\

    \bottomrule

    \end{tabular}
%     \label{tab:requisiti_non_funz}

\end{center}

\section{User Experience e usabilità}

Oltre al consolidamento dei requisiti, le fasi di analisi e test (racchiuse in ogni iterazione della metodologia Agile) hanno permesso una graduale evoluzione del progetto, partito da uno stadio prototipale, anche sotto il punto di vista dell'usabilità e della user experience.

I test e i collaudi sono stati eseguiti presentando il progetto su dispositivi Apple iPad a personale selezionato (per policy interne i collaudi sono avvenuti con personale interno all'azienda cliente) e sono stati discussi, raccolti e valutati i feedback ottenuti durante l'utilizzo del software.
Questo approccio ha consentito l'immediata valutazione della qualità del software prodotto e di concentrarsi immediatamente sull'iterazione successiva per migliorarlo.

\subsection{iOS Human Interface Guidelines}

Le \emph{GUI} (graphic user interface) sono state realizzate considerando le linee guida fornite da Apple nelle \emph{Human Interface Guidelines}. Di seguito sono elencate alcune delle linee chiave adottate:

\begin{itemize}
 \item Integrità estetica: rappresenta quanto l'apparenza di un'interfaccia è coerente con le azioni e le funzionalità che offre
 \item Consistenza: permettere all'utente di trasferire le proprie abilità e conoscenze da un'interfaccia di un'applicazione all'altra 
 \item Feedback: fornire un feedback all'utente permette di avere una risposta immediata di un'azione intrapresa sull'interfaccia 
\end{itemize}

Oltre ai componenti standard offerti dall'\emph{SDK} iOS sono stati realizzati dei componenti grafici custom che hanno migliorato l'esperienza d'uso rispettando al tempo stesso le linee guida e il know how medio posseduto dalle persone che utilizzano dispositivi mobili.  

\subsection{Esempi di GUI e raffinamenti successivi}

\subsubsection{Colori}
Inizialmente si era scelto di utilizzare un tema di colori focalizzato sullo stesso colore del brand del committente e degli altri prodotti precedentemente realizzati(esempio: sito web, app smartphone, ecc\dots).

I test con gli utenti hanno però evidenziato che temi di colori troppo scuri possono risultare pesanti alla vista e percepiti come difficili da leggere. Si è scelto quindi di utilizzare un tema di colori chiaro, con gli elementi più importanti in contrasto di colore in modo tale da indurre un focus immediato e guidare l'utente durante le operazioni.

\begin{figure}[!htbp]
\centering
\includegraphics[scale=0.30]{ux/bonificonero.png}
\caption{Tema iniziale nero per la sezione bonifico}
\end{figure}

\begin{figure}[!htbp]
\centering
\includegraphics[scale=0.30]{ux/bonificogrigio.png}
\caption{Tema chiaro per il bonifico nella sua versione finale, si noti il pulsante \emph{conferma} in risalto rispetto altri elementi della GUI}
\end{figure}

\begin{figure}[!htbp]
\centering
\includegraphics[scale=0.30]{ux/timeline3.png}
\caption{Tema verde per il background della timeline}
\end{figure}
\begin{figure}[!htbp]
\centering
\includegraphics[scale=0.30]{ux/timelinebianca.png}
\caption{Tema chiaro per il background della timeline e utilizzo del colore rosso e grigio per dare risalto ai saldi negativi e positivi dei movimenti}
\end{figure}

\subsubsection{Suggerimenti}

Dai test effettuati è emersa la necessita di aiutare l'utente attraverso suggerimenti (\emph{hint}) visivi posti in particolari punti delle interfacce. L'utilizzo volutamente moderato di suggerimenti ha aumentato e migliorato la \emph{discoverability}. 
\begin{figure}[!htbp]
\centering
\includegraphics[scale=0.30]{ux/timelinenohint.png}
\caption{L'utente premeva il pulsante \emph{menu} per cercare di tornare alla schemata dei conti}
\end{figure}
\begin{figure}[!htbp]
\centering
\includegraphics[scale=0.30]{ux/timelinehint.png}
\caption{Con una freccia che rappresentasse il significato di \emph{indietro} si è messo in risalto la relativa azione}
\end{figure}

\newpage
\subsubsection{Feedback}
Per indicare che, in risposta ad una azione dell'utente, è avvenuto un evento nel sistema  si è scelto di utilizzare feedback visivi e diversificati in base al tipo di azione intrapresa. 
Di seguito sono riportati alcuni dei feedback utilizzati all'interno dell'applicazione:

\begin{itemize}
 \item componenti che indicano il caricamento durante task di lunga durata (come ad esempio durante le chiamate http ai servizi, figura \ref{fig:loading})
 \item l'inserimento di animazioni per passare da un'interfacca ad un'altra dopo l'azione eseguita dall'utente (figura \ref{fig:animation})
 \item elementi grafici che cambiano il loro stato in risposta a eventi particolari (figura \ref{fig:copy})

\end{itemize}


\begin{figure}[htbp]
\centering
\includegraphics[scale=0.30]{ux/loading.png}
\caption{Esempi di loading view durante la chiamata ai servizi}
\label{fig:loading}
\end{figure}

\begin{figure}[htbp]
\centering
\includegraphics[scale=0.30]{ux/animazione.png}
\caption{Fotogramma preso durante un'animazione rappresentante il cambio di interfaccia e del relativo contesto (colori, immagini, contenuti)}
\label{fig:animation}
\end{figure}

\begin{figure}[htbp]
\centering
\includegraphics[scale=0.30]{ux/copia.png}
\caption{Dopo l'azione di copia il relativo pulsante cambia per qualche frazione di secondo il suo contenuto in \emph{copiato!}}
\label{fig:copy}
\end{figure}

\subsubsection{Componenti personalizzati}
L'utilizzo di componenti grafici personalizzati ha permesso di ottenere interfacce grafiche più ricche di funzioni e al tempo stesso intuitive e facili da usare. La figura \ref{fig:selettoreC} rappresenta un componente grafico che ha la funzione di poter selezionare attraverso un tap sui pulsanti o attraverso la gesture swipe un elemento in una lista di oggetti (in questo caso di conti); tale componente è stato realizzato come libreria esterna, e quindi riusabile in diversi contesti della stessa applicazione.
\begin{figure}[htbp]
\centering
\includegraphics[scale=0.50]{ux/selettore.png}

\caption{Il componente custom \emph{selettore conti}}
\label{fig:selettoreC}
\end{figure}

Un altro esempio di componente personalizzato sono i tasti di azione rapida accessibili effettuando la gesture \emph{swipe} sulle righe della \emph{Message box} interna all'applicazione.
In questo caso si è ricreato lo stesso design e si è mantenuta la stessa interazione che presenta l'applicazione \emph{Mail} nativa di iOS 7, permettendo così all'utente di riutilizzare le stesse abilità acquisite nell'utilizzo di un software originale Apple.
\begin{figure}[!htbp]
\centering
\includegraphics[scale=0.50]{ux/messageBox.png}

\caption{Dettaglio delle funzionalità della \emph{Message box}}
\label{fig:selettore}
\end{figure}
\subsubsection{Help}
Per permettere un facile e rapido utilizzo dell'applicazione a tutti i possibili utenti, che spaziano dall'utilizzatore abituale di dispositivi e applicazioni mobili a quelli che si avvicinano per la prima volta alle nuove tecnologie, si è deciso di utilizzare al primo avvio dell'app alcune schermate di aiuto (figura \ref{fig:help}) con la finalità di descrivere le principali  e le meno intuitive funzionalità del software.

Le fasi di test hanno evidenziato l'effettiva necessità di tale scelta che ha portato al miglioramento della discoverability, soprattutto nei primi utilizzi dell'app da parte dell'utente.

\begin{figure}[!htbp]
\centering
\includegraphics[scale=0.20]{ux/help.jpg}

\caption{Schermata di help per descrivere le funzionalità dell'interfaccia di riepilogo di conti e carte}
\label{fig:help}
\end{figure}


\subsubsection{Prestazioni}
Per rendere più gradevole la user experience si è deciso di eseguire il \emph{caching}\footnote{Eseguire il salvataggio in ram o in memoria secondaria dei dati di frequente utilizzo, in modo tale da renderli disponibili per accessi futuri.} di alcuni dei dati scaricati dalla rete (come ad esempio la lista dei prodotti, o dati relativi al gestore, ecc\dots). Questa scelta ha quindi permesso di ridurre i tempi di accesso ad alcuni servizi migliorando le prestazioni del software in generale e l'interazione dell'utente con lo stesso.  